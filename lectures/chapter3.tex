\chapter{Základy řešení parciálních diferenciálních rovnic}

Zavzpomíname-li na základní kurzy matematické analýzy, jistě si vzpomene, jak 
nesnadné je analytické řešení parciálních diferenciálních rovnic, pokud vůbec lze 
řešení najít. Když pomineme učebnicové příklady (za zmínku stojí například vlnová 
rovnice), stojíme většinou před neřešitelným problémem. Naštestí pro nás ale ne pro 
fyziku obecně, pomocnou ruku nám podá numerické řešení problému a síla současné 
výpočetní techniky~-- počítače. Jak ale na to? Jak převést rovnici, kterou jsme 
dostali aplikací fyzikálních zákonů pro konkrétní problém do řeči čísel? 
Následující kapitola se vám pokusí v tom udělat trochu jasněji.

\section{Metoda konečných diferencí}

Jednou z nejpoužívanějších metod, která je zároveň vhodná pro názornou ilustraci, 
je {\it metoda konečných diferencí}. Nejedná se o nic jiného než diskrétní 
reprezentaci patřičných proměnných, funkcí a derivací definovaného problému. Zní to 
složitě ale ve skutečnosti je to velmi jednoduché a vše co k tomu budeme potřebovat 
je znalost Taylorova rozvoje funkce. Názorně si to ilustrujeme na jednoduché 
rovnici

\begin{equation}
\frac{\partial u}
	 {\partial t} + v \frac{\partial u}
	                       {\partial x} = D \frac{\partial^2 u}
	                                             {\partial 	x^2}
\label{rovnice}
\end{equation}

zahrnující v sobě jak difúzní $D\frac{\partial^2 u}
                                      {\partial x^2}$,
tak advekční člen $v\frac{\partial u}
                          {\partial x}$.                          
Funkce $u(x, t)$ nám udává $x$-ovou hodnotu rychlosti. Numerický přístup řešení 
této rovnice spočívá v reprezentaci $u$ souborem diskrétních hodnot $u_{i}$ v 
bodech diskrétní sítě

$$
x_0, x_1, x_2, \dots, x_i, \dots, x_N \quad (x_0 < x_1 < x_2 \dots < x_N)
$$

Na první pohled je patrné, že s rostoucím počtem bodů sítě, se bude naše 
reprezentace blížit skutečné, které bychom dosáhli pro $N = \infty$.

\subsection*{Prostorové derivace}

S touto reprezentací se můžeme dále pustit do aproximací prostorových derivací. K 
tomu využijeme Taylorova rozvoje okolo bodu $u_i$ pro hodnotu v bodě $u_{i+1}$. 
Směle můžeme psát

\begin{equation}
u_{i+1} = u_{i} +
  \left( \frac{\partial     u}{\partial   x} \right)_{i} \Delta x
+ \left( \frac{{\partial}^2 u}{\partial x^2} \right)_{i} \frac{(\Delta x)^2}{2}
+ \left( \frac{{\partial}^3 u}{\partial x^3} \right)_{i} \frac{(\Delta x)^3}{6}
+ \dots
\label{FF}
\end{equation}

Obdobně pro hodnotu $u_{i-1}$

\begin{equation}
u_{i-1} = u_{i}
- \left( \frac{\partial     u}{\partial   x} \right)_{i} \Delta x 
+ \left( \frac{{\partial}^2 u}{\partial x^2} \right)_{i} \frac{(\Delta x)^2}{2}
- \left( \frac{{\partial}^3 u}{\partial x^3} \right)_{i}\frac{(\Delta x)^3}{6}
+ \dots
\label{FB}
\end{equation}

Z Taylorova rozvoje můžeme jednoduše vyjádřit vztah pro derivaci v daném bodě $i$ 
pomocí hodnot např. $u_{i}$ a $u_{i+1}$ (nebo také $u_{i}$ a $u_{i-1}$)

\begin{equation}
\left( \frac{\partial u}{\partial x} \right)_{i}
= \frac{u_{i+1}-u_{i}}{\Delta x}
- \left( \frac{{\partial}^2 u}{\partial x^2} \right) \frac{\Delta x}{2}
- \left( \frac{{\partial}^3 u}{\partial x^3} \right) \frac{(\Delta x)^2}{6}
+ \dots
\end{equation}

Tím se dostaváme k určení prvních derivací podle prostorové souřadnice, rozlišujeme

\begin{tcolorbox}[title=Diference I. řádu - první derivace]
\begin{itemize}
    \item{ {\bf Prostorové diference (vpřed) FDS}
\begin{equation}
  \left( \frac{\partial u}{\partial x}\right)_{i}
= \frac{u_{i+1}-u_{i}}{\Delta x}+O(\Delta x)
\label{FD}
\end{equation}}
    \item{ {\bf Prostorové diference (dozadu) BDS}
\begin{equation}
  \left( \frac{\partial u}{\partial x}\right)_{i}
= \frac{u_{i}-u_{i-1}}{\Delta x} + O(\Delta x)
\label{BD}
\end{equation}}
\end{itemize}
\end{tcolorbox}

Vidíme tak, že první prostorové derivace naší funkce můžeme jednoduše vyjádřit ze 
znalostí hodnot funkce $u$ v diskrétních bodech $i-1,i,i+1$, v závislosti na 
zvoleném způsobu (\ref{FD}) resp. (\ref{BD}). Chyba, které se při této aproximaci 
dopouštíme, je prvního řádu, jak je patrné z Taylorova rozvoje.

V mnoha případech není však metoda prvního řádu dostatečná, je třeba použít 
přesnější metody, tedy druhého řádu. Odečtením rovnic (\ref{FF}) a (\ref{FB}) pro 
diferenci vzad a vpřed s Taylorovym rozvojem dostaneme výraz pro středovou 
diferenci (CD) s přesností druhého řádu \\

\begin{tcolorbox}[title = Diference II.řádu - první derivace]
\begin{itemize}
\item{{\bf Prostorové diference (centrální) CDS}
\begin{equation}
  \left( \frac{\partial u}{\partial x}\right)_{i}
= \frac{u_{i+1}-u_{i-1}}{2\Delta x} + O(\Delta x)^2
\label{CD}
\end{equation}}
\end{itemize}
\end{tcolorbox}

Nic nám již nebrání, abychom vyjádřili i druhé derivace. Stačí nám k tomu sečíst 
rovnice (\ref{FF}) a (\ref{FB}) a po úpravě dostáváme pro druhou derivaci 
diferenční vztah s přesností třetího řádu

\begin{tcolorbox}[title = Diference I. řádu - druhá derivace]
{\bf Diferenční vztah pro druhou derivaci}
\begin{equation}
  \left( \frac{{\partial}^2 u}{\partial x^2}\right)_{i}
= \frac{u_{i+1} - 2 u_i + u_{i-1}}{(\Delta x)^2}
\label{SecDer}
\end{equation}
\end{tcolorbox}

\subsection*{Časové derivace}

Obdobně budeme postupovat při určování časové derivace, nicméně je třeba přiznat, 
že se nám situace trochu komplikuje. Příčina změny je patrná z předchozích vzorců, 
rozdíl spočívá ve znalosti prostorových hodnot $u_i$ v daném časovém okamžiku. 
Prostorové derivace můžeme vyjádřit velmi snadno, pro časovou derivaci je třeba 
uvážit, že známé hodnoty funkce $u_i(t)$ jsou pouze ty současné (přítomnost) a z 
předchozích kroků (minulost). Hodnoty následující nám známé nejsou a je třeba je 
určit. Pro lepší pochopení si nejprve formálně vyjádříme časovou derivaci z rovnice 
(\ref{rovnice})

\begin{equation}
\frac{\partial u}{\partial t} = h(u, x, t)
\label{cas_der}
\end{equation}

Dále budeme postupovat jako pro prostorové derivace, nejprve diskretizujeme čas na 
jednotlivé kroky

$$
t_0, t_1, t_2, \dots, t_{n}, \dots, t_{M} \quad (t_0 < t_1 < t_2 \dots < t_{M})
$$

a určíme jednotlivé derivace podle stejného receptu

\begin{tcolorbox}[title=Časové diference I. řádu]
\begin{itemize}
\item{{\bf  (vpřed) FDT}
\begin{equation}
  \left( \frac{\partial u}{\partial t} \right)_{i}^{n}
= \frac{u_{i}^{n+1} - u_{i}^{n}}{\Delta t} + O(\Delta t)
\label{FD}
\end{equation}}
\end{itemize}
\end{tcolorbox}

Díky počatečním podmínkám známe v daném počátečním okamžiku všechny hodnoty $u_i$, 
pro $i=1,\dots,N$. Jak je naznačeno indexem u derivace na pravé straně, v tomto 
případě použijeme známé hodnoty z času $t=n$. Pro časovou derivaci (\ref{cas_der}) 
platí

\begin{eqnarray}
\frac{u_{i}^{n+1} - u_{i}^{n}}{\Delta t} = h(u^{n}, x^{n}, t)
\\
u_{i}^{n+1} = u_{i}^n + h^{n} \Delta{t} + O(\Delta x)
\end{eqnarray}

Tento hojně využívaný přístup je označován jako explicitní metoda konečných 
diferencí v čase směrem vpřed (FFTD). Obdobně pokud využijeme druhý způsob 
vyjádření derivace, tedy

\begin{tcolorbox}[title=Časové diference I. řádu]
\begin{itemize}
    \item{{\bf Časové diference (dozadu) BDT}
\begin{equation}
\left( \frac{\partial u}{\partial t} \right)_{i}^{n}
= \frac{u_{i}^{n}-u_{i}^{n-1}}{\Delta t} + O(\Delta t) = \|^{n+1}\, 
  \frac{u_{i}^{n+1}-u_{i}^{n}}{\Delta t} + O(\Delta t) 
\label{BD}
\end{equation}}
\end{itemize}
\end{tcolorbox}

Rozdíl oproti předchozímu způsobu vyjádření spočívá v použití neznámých hodnot z 
času $t = n+1$. Pro časovou derivaci (\ref{cas_der}) tak plyne

\begin{eqnarray}
\frac{u_{i}^{n+1}-u_{i}^n}{\Delta t} = h(u^{n+1}, x^{n+1}, t)
\\
u_{i}^{n+1} = u_{i}^n+h^{n+1}\Delta{t} + O(\Delta x),
\end{eqnarray}

která je označována jako implicitní metoda (BFTD). Zcela analogicky pak dostaneme 
obdobné vyjádření (taktéž implicitní) pokud použijeme centrální diferenční schéma 
(s přesností druhého řádu)

\begin{equation}
u_{i}^{n+1} = u_{i}^n + \frac{(h^{n+1} + h^{n})}{2} \Delta{t} + O(\Delta x)^2
\end{equation}

O tom, který ze způsobu vyjádření je lepší lze vést dlouhé diskuze, každý z těchto 
způsobů má své výhody a nevýhody, ať už z hlediska výpočetních nároků a nebo 
stability.

Nyní je již vše připraveno pro převod (\ref{rovnice}) do diskrétního schématu. Pro 
náš ilustrační případ zvolme pro časovou derivaci FD diferenci a CD pro advekční 
člen, druhá derivace je dána vztahem (\ref{SecDer}). Dostáváme výraz

\begin{equation}
     \frac{u_{i}^{n+1} - u_i^{n}}{\Delta t}
= - v \frac{u_{i+1}^n - u_{i-1}^n}{2\Delta x}
  + D \frac{u_{i+1}^n - 2 u_i^n+u_{i-1}^n}{(\Delta x)^2}
\end{equation}

který ještě přeuspořádáme do podoby numerického schématu (označovaného jako FTCS), 
abychom neznámé veličiny měli na leve straně a známe na pravé straně

\begin{equation}
u_{i}^{n+1} 
= u_i^{n} 
- \frac{1}{2} \frac{v \Delta t}{\Delta  x}(u_{i+1}^n - u_{i-1}^n)
+ D \frac{\Delta t}{(\Delta x)^2}(u_{i+1}^n - 2 u_{i}^{n} + u_{i-1}^n)
\label{simplefinal}
\end{equation}

Je patrné, že způsob, jakým lze původní rovnici přepsat (\ref{rovnice}) do 
diferenčního schématu není jednoznačný, máme nepřeberné množství možností 
explicitních i implicitních způsobů, které se navzájem liší výpočetní náročností, 
složitostí i stabilitou. V následujících odstavcích se seznámíme s 
nejpouživanějšími metodami.

\section{Nejjednodušší numerická schémata}

\subsection*{Laxova metoda}

Je jednou z nejjednoduších metod, hojně využivanou v numerické hydrodynamice. 
Dostaneme ji nahrazením $u_i^{n}$ v časové derivaci v rovnici (\ref{simplefinal}) 
průměrnou hodnotou určenou z jejich sousedů

$$
u_{i}^{n} \approx \frac{(u_{i+1}^{n} + u_{i-1}^n)}{2}
$$

Obdržíme Laxovo diferenční schéma

\begin{equation}
u_{i}^{n+1} 
=   \frac{1}{2}(u_{i+1}^{n}+u_{i-1}^{n})
-   \frac{1}{2}\frac{v\Delta t}{\Delta x}(u_{i+1}^n-u_{i-1}^n)
+ D \frac{\Delta t}{(\Delta x)^2}(u_{i+1}^n - 2 u_{i}^{n} + u_{i-1}^n)
\end{equation}

\subsection*{Upwind schéma }

Zvolené schéma volí jiný přístup, respektuje fyzikální podstatu problému, jinými 
slovy respektruje směr šíření proudu (tedy informace) v advekčním členu rovnice 
(\ref{rovnice}). Místo použití CD prostorové diference pro advekční člen se použije 
buď FD pro případ záporné advekční rychlosti$ v < 0$, nebo BD v případě kladné 
advekční rychlosti $v>0$. Výsledné schéma

\begin{eqnarray}
  u_{i}^{n+1}
= u_{i}^{n}
+ D\frac{\Delta t}{(\Delta x)^2}(u_{i+1}^n - 2 u_{i}^{n} + u_{i-1}^n)
-
\begin{cases}
    \frac{v\Delta t}{\Delta x}(u_{i}^n - u_{i-1}^n) \quad v > 0
    \\
    \frac{v\Delta t}{\Delta x}(u_{i+1}^n - u_{i}^n) \quad v < 0
\end{cases}
\label{upwind}
\end{eqnarray}

\subsection*{Crank-Nicholson}

Nemusíme se však omezit pouze na explicitní metody. Znamým příkladem implicitní 
metody je Crankovo-Nicholsonovo schéma, kde pro výpočet použijeme hodnoty v čase 
$t$ a $t+\Delta$ a to tak, že pro stanovení výsledné hodnoty použijeme průměr obou 
hodnot. Rovnici přepíšeme (\ref{rovnice}),

\begin{eqnarray}
  u_{i}^{n+1}
= u_i^{n}
-   \frac{1}{2}\frac{v\Delta t}{\Delta x}
    \left( \frac{1}{2}(u_{i+1}^{n+1} + u_{i+1}^{n})
-   \frac{1}{2}(u_{i-1}^{n+1} + u_{i-1}^{n}) \right)
\\
+ D \frac{\Delta t}{(\Delta x)^2}
    \left( \frac{1}{2}(u_{i+1}^{n+1} + u_{i+1}^n)
-   \frac{1}{2}(2 u_{i}^{n+1} + 2u_{i}^n)
+   \frac{1}{2}(u_{i-1}^{n+1} + u_{i-1}^n) \right)
\end{eqnarray}

a následně ještě schéma upravíme

\begin{eqnarray}
u_{i}^{n+1} & = & u_i^{n}
- \frac{1}{4}\frac{v\Delta t}{\Delta x}
  (u_{i+1}^{n+1} - u_{i-1}^{n+1} + u_{i+1}^{n} - u_{i-1}^n)
  \\
  \nonumber & + & D \frac{\Delta t}{(\Delta x)^2}
  \frac{1}{2}(u_{i+1}^{n+1}-2u_{i}^{n+1}+u_{i-1}^{n+1}+u_{i+1}^n-2u_i^n+u_{i-1}^n)
\label{crank_1}
\end{eqnarray}

Vidíme, že na rozdíl od předchozích případů nám neznámé hodnoty funkce 
$u_j^{n+1}$ v diskrétních bodech $j$ vystupují i na pravé straně. Toto schéma 
vede v tomto případě k řešení soustavy (ne)linearních rovnic (v případě nelinearity 
nutno linearizovat - například Burgersova rovnice). Pokud však $v=0$, dostáváme 
difůzní rovnici a lineární soustavu. Soustavu rovnic nejprve přepíšeme tak, že 
všechny neznáme veličiny převedeme na levou stranu. Označme

\begin{eqnarray}
\sigma = \frac{D\Delta t}{2\Delta x^2} \\
\rho   = \frac{1}{4}\frac{v\Delta t}{\Delta x}
\end{eqnarray}

pak můžeme psát pro rovnice (\ref{crank_1})

\begin{equation}
  u_{i-1}^{n+1}(-\sigma-\rho) + u_i^{n+1}(1+2\sigma) + u_{i+1}^{n+1}
= u_{i-1}^n(\sigma+\rho) + u_i^{n}(1-2\sigma) + u_{i+1}^{n}(\sigma-\rho)
\end{equation}

V maticovém formalismu můžeme rovnice přepsat následovně s použitím substituce
$A = - (\sigma+\rho)$, $B = (1+2\sigma)$, $C = (\rho-\sigma)$

\begin{equation}
\begin{pmatrix}
B & C & 0 & \cdots & \cdots & 0 \\
A & B & C & \ddots & \ddots & \vdots \\
0 & A & B & \ddots & \ddots & \vdots \\
\vdots & \ddots & \ddots & \ddots & \ddots & 0\\
\vdots & \ddots & \ddots & A & B & C \\
0 & \cdots & \cdots & 0 & A & B 
\end{pmatrix}
\begin{pmatrix}
u_1 \\ u_2 \\ \vdots \\ \\ \vdots \\ u_{M}
\end{pmatrix}
=
\begin{pmatrix}
R_1 \\ R_2 \\ \vdots \\ \\ \vdots \\ R_M
\end{pmatrix}
\end{equation}

Vektor pravých stran je dán vztahem

\begin{equation}
R_i = u_{i-1}^n(\sigma+\rho) + u_i^n(1-2\sigma)
    + u_{i+1}^n(\sigma-\rho) \quad (i \neq 1,M)
\end{equation}

Pro případ hodnoty $R_0, R_M$, je nutné si uvědomit, že hodnota funkce 
$u_{0}^{n+1}$ a $u_{M+1}^{n+1}$ je známa díky počatečním podmínkám, konkrétně

\begin{eqnarray}
u_{0}^{n+1} & = & u_{0}^{n} \\
u_{M+1}^{n+1} & = & u_{M+1}^{n} 
\end{eqnarray}

\section{Numerické testy}
